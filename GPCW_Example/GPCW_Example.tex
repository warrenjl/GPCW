% Options for packages loaded elsewhere
\PassOptionsToPackage{unicode}{hyperref}
\PassOptionsToPackage{hyphens}{url}
%
\documentclass[
]{article}
\usepackage{amsmath,amssymb}
\usepackage{lmodern}
\usepackage{ifxetex,ifluatex}
\ifnum 0\ifxetex 1\fi\ifluatex 1\fi=0 % if pdftex
  \usepackage[T1]{fontenc}
  \usepackage[utf8]{inputenc}
  \usepackage{textcomp} % provide euro and other symbols
\else % if luatex or xetex
  \usepackage{unicode-math}
  \defaultfontfeatures{Scale=MatchLowercase}
  \defaultfontfeatures[\rmfamily]{Ligatures=TeX,Scale=1}
\fi
% Use upquote if available, for straight quotes in verbatim environments
\IfFileExists{upquote.sty}{\usepackage{upquote}}{}
\IfFileExists{microtype.sty}{% use microtype if available
  \usepackage[]{microtype}
  \UseMicrotypeSet[protrusion]{basicmath} % disable protrusion for tt fonts
}{}
\makeatletter
\@ifundefined{KOMAClassName}{% if non-KOMA class
  \IfFileExists{parskip.sty}{%
    \usepackage{parskip}
  }{% else
    \setlength{\parindent}{0pt}
    \setlength{\parskip}{6pt plus 2pt minus 1pt}}
}{% if KOMA class
  \KOMAoptions{parskip=half}}
\makeatother
\usepackage{xcolor}
\IfFileExists{xurl.sty}{\usepackage{xurl}}{} % add URL line breaks if available
\IfFileExists{bookmark.sty}{\usepackage{bookmark}}{\usepackage{hyperref}}
\hypersetup{
  hidelinks,
  pdfcreator={LaTeX via pandoc}}
\urlstyle{same} % disable monospaced font for URLs
\usepackage[margin=1in]{geometry}
\usepackage{color}
\usepackage{fancyvrb}
\newcommand{\VerbBar}{|}
\newcommand{\VERB}{\Verb[commandchars=\\\{\}]}
\DefineVerbatimEnvironment{Highlighting}{Verbatim}{commandchars=\\\{\}}
% Add ',fontsize=\small' for more characters per line
\usepackage{framed}
\definecolor{shadecolor}{RGB}{248,248,248}
\newenvironment{Shaded}{\begin{snugshade}}{\end{snugshade}}
\newcommand{\AlertTok}[1]{\textcolor[rgb]{0.94,0.16,0.16}{#1}}
\newcommand{\AnnotationTok}[1]{\textcolor[rgb]{0.56,0.35,0.01}{\textbf{\textit{#1}}}}
\newcommand{\AttributeTok}[1]{\textcolor[rgb]{0.77,0.63,0.00}{#1}}
\newcommand{\BaseNTok}[1]{\textcolor[rgb]{0.00,0.00,0.81}{#1}}
\newcommand{\BuiltInTok}[1]{#1}
\newcommand{\CharTok}[1]{\textcolor[rgb]{0.31,0.60,0.02}{#1}}
\newcommand{\CommentTok}[1]{\textcolor[rgb]{0.56,0.35,0.01}{\textit{#1}}}
\newcommand{\CommentVarTok}[1]{\textcolor[rgb]{0.56,0.35,0.01}{\textbf{\textit{#1}}}}
\newcommand{\ConstantTok}[1]{\textcolor[rgb]{0.00,0.00,0.00}{#1}}
\newcommand{\ControlFlowTok}[1]{\textcolor[rgb]{0.13,0.29,0.53}{\textbf{#1}}}
\newcommand{\DataTypeTok}[1]{\textcolor[rgb]{0.13,0.29,0.53}{#1}}
\newcommand{\DecValTok}[1]{\textcolor[rgb]{0.00,0.00,0.81}{#1}}
\newcommand{\DocumentationTok}[1]{\textcolor[rgb]{0.56,0.35,0.01}{\textbf{\textit{#1}}}}
\newcommand{\ErrorTok}[1]{\textcolor[rgb]{0.64,0.00,0.00}{\textbf{#1}}}
\newcommand{\ExtensionTok}[1]{#1}
\newcommand{\FloatTok}[1]{\textcolor[rgb]{0.00,0.00,0.81}{#1}}
\newcommand{\FunctionTok}[1]{\textcolor[rgb]{0.00,0.00,0.00}{#1}}
\newcommand{\ImportTok}[1]{#1}
\newcommand{\InformationTok}[1]{\textcolor[rgb]{0.56,0.35,0.01}{\textbf{\textit{#1}}}}
\newcommand{\KeywordTok}[1]{\textcolor[rgb]{0.13,0.29,0.53}{\textbf{#1}}}
\newcommand{\NormalTok}[1]{#1}
\newcommand{\OperatorTok}[1]{\textcolor[rgb]{0.81,0.36,0.00}{\textbf{#1}}}
\newcommand{\OtherTok}[1]{\textcolor[rgb]{0.56,0.35,0.01}{#1}}
\newcommand{\PreprocessorTok}[1]{\textcolor[rgb]{0.56,0.35,0.01}{\textit{#1}}}
\newcommand{\RegionMarkerTok}[1]{#1}
\newcommand{\SpecialCharTok}[1]{\textcolor[rgb]{0.00,0.00,0.00}{#1}}
\newcommand{\SpecialStringTok}[1]{\textcolor[rgb]{0.31,0.60,0.02}{#1}}
\newcommand{\StringTok}[1]{\textcolor[rgb]{0.31,0.60,0.02}{#1}}
\newcommand{\VariableTok}[1]{\textcolor[rgb]{0.00,0.00,0.00}{#1}}
\newcommand{\VerbatimStringTok}[1]{\textcolor[rgb]{0.31,0.60,0.02}{#1}}
\newcommand{\WarningTok}[1]{\textcolor[rgb]{0.56,0.35,0.01}{\textbf{\textit{#1}}}}
\usepackage{graphicx}
\makeatletter
\def\maxwidth{\ifdim\Gin@nat@width>\linewidth\linewidth\else\Gin@nat@width\fi}
\def\maxheight{\ifdim\Gin@nat@height>\textheight\textheight\else\Gin@nat@height\fi}
\makeatother
% Scale images if necessary, so that they will not overflow the page
% margins by default, and it is still possible to overwrite the defaults
% using explicit options in \includegraphics[width, height, ...]{}
\setkeys{Gin}{width=\maxwidth,height=\maxheight,keepaspectratio}
% Set default figure placement to htbp
\makeatletter
\def\fps@figure{htbp}
\makeatother
\setlength{\emergencystretch}{3em} % prevent overfull lines
\providecommand{\tightlist}{%
  \setlength{\itemsep}{0pt}\setlength{\parskip}{0pt}}
\setcounter{secnumdepth}{-\maxdimen} % remove section numbering
\ifluatex
  \usepackage{selnolig}  % disable illegal ligatures
\fi

\author{}
\date{\vspace{-2.5em}}

\begin{document}

\hypertarget{gpcw-gaussian-process-model-for-critical-window-estimation}{%
\section{GPCW: Gaussian Process Model for Critical Window
Estimation}\label{gpcw-gaussian-process-model-for-critical-window-estimation}}

\hypertarget{gpcw_example}{%
\subsection{GPCW\_Example}\label{gpcw_example}}

{[}1{]} Simulate data from the proposed model:

\begin{itemize}
\tightlist
\item
  Setting the reproducibility seed and initializing packages for data
  simulation:
\end{itemize}

\begin{Shaded}
\begin{Highlighting}[]
\FunctionTok{set.seed}\NormalTok{(}\DecValTok{8453}\NormalTok{)}

\FunctionTok{library}\NormalTok{(GPCW)  }
\FunctionTok{library}\NormalTok{(mnormt)  }\CommentTok{\#Multivariate normal distribution}
\FunctionTok{library}\NormalTok{(boot)  }\CommentTok{\#Inverse logit transformation}
\end{Highlighting}
\end{Shaded}

\begin{itemize}
\tightlist
\item
  Setting the global data values:
\end{itemize}

\begin{Shaded}
\begin{Highlighting}[]
\NormalTok{n}\OtherTok{\textless{}{-}}\DecValTok{5000}  \CommentTok{\#Sample size}
\NormalTok{m}\OtherTok{\textless{}{-}}\DecValTok{36}  \CommentTok{\#Number of exposure time periods}
\NormalTok{x}\OtherTok{\textless{}{-}}\FunctionTok{matrix}\NormalTok{(}\DecValTok{1}\NormalTok{, }
          \AttributeTok{nrow=}\NormalTok{n, }
          \AttributeTok{ncol=}\DecValTok{1}\NormalTok{)  }\CommentTok{\#Covariate design matrix}
\NormalTok{z}\OtherTok{\textless{}{-}}\FunctionTok{matrix}\NormalTok{(}\FunctionTok{rnorm}\NormalTok{(}\AttributeTok{n=}\NormalTok{(n}\SpecialCharTok{*}\NormalTok{m)), }
          \AttributeTok{nrow=}\NormalTok{n, }
          \AttributeTok{ncol=}\NormalTok{m)  }\CommentTok{\#Exposure design matrix}

\ControlFlowTok{for}\NormalTok{(j }\ControlFlowTok{in} \DecValTok{1}\SpecialCharTok{:}\NormalTok{m)\{}
\NormalTok{   z[,j]}\OtherTok{\textless{}{-}}\NormalTok{(z[,j] }\SpecialCharTok{{-}} \FunctionTok{median}\NormalTok{(z[,j]))}\SpecialCharTok{/}\FunctionTok{IQR}\NormalTok{(z[,j])  }\CommentTok{\#Data standardization (interquartile range)}
\NormalTok{   \}}
\end{Highlighting}
\end{Shaded}

\begin{itemize}
\tightlist
\item
  Setting the values for the statistical model parameters:
\end{itemize}

\begin{Shaded}
\begin{Highlighting}[]
\NormalTok{beta\_true}\OtherTok{\textless{}{-}} \SpecialCharTok{{-}}\FloatTok{0.30}
\NormalTok{sigma2\_theta\_true}\OtherTok{\textless{}{-}}\FloatTok{0.50}
\NormalTok{phi\_true}\OtherTok{\textless{}{-}}\FloatTok{0.01}
\NormalTok{Sigma\_true}\OtherTok{\textless{}{-}}\NormalTok{sigma2\_theta\_true}\SpecialCharTok{*}\FunctionTok{chol2inv}\NormalTok{(}\FunctionTok{chol}\NormalTok{(}\FunctionTok{temporal\_corr\_fun}\NormalTok{(m, }
\NormalTok{                                                              phi\_true)[[}\DecValTok{1}\NormalTok{]]))}
\NormalTok{theta\_true}\OtherTok{\textless{}{-}}\FunctionTok{rmnorm}\NormalTok{(}\AttributeTok{n=}\DecValTok{1}\NormalTok{, }
                   \AttributeTok{mean=}\FunctionTok{rep}\NormalTok{(}\DecValTok{0}\NormalTok{, }\AttributeTok{times=}\NormalTok{m), }
                   \AttributeTok{varcov=}\NormalTok{Sigma\_true)}
\NormalTok{theta\_true}\OtherTok{\textless{}{-}}\NormalTok{theta\_true }\SpecialCharTok{{-}} \FunctionTok{mean}\NormalTok{(theta\_true)}
\NormalTok{logit\_p\_true}\OtherTok{\textless{}{-}}\NormalTok{x}\SpecialCharTok{\%*\%}\NormalTok{beta\_true }\SpecialCharTok{+} 
\NormalTok{              z}\SpecialCharTok{\%*\%}\NormalTok{theta\_true}
\NormalTok{probs\_true}\OtherTok{\textless{}{-}}\FunctionTok{inv.logit}\NormalTok{(logit\_p\_true)}
\FunctionTok{hist}\NormalTok{(probs\_true)}
\end{Highlighting}
\end{Shaded}

\includegraphics{GPCW_Example_files/figure-latex/unnamed-chunk-3-1.pdf}

\begin{itemize}
\tightlist
\item
  Simulating the analysis dataset:
\end{itemize}

\begin{Shaded}
\begin{Highlighting}[]
\NormalTok{y}\OtherTok{\textless{}{-}}\FunctionTok{rbinom}\NormalTok{(}\AttributeTok{n=}\NormalTok{n, }
          \AttributeTok{size=}\DecValTok{1}\NormalTok{, }
          \AttributeTok{prob=}\NormalTok{probs\_true)}
\end{Highlighting}
\end{Shaded}

{[}2{]} Fit GPCW to estimate critical windows of susceptibility:

\begin{Shaded}
\begin{Highlighting}[]
\NormalTok{results}\OtherTok{\textless{}{-}}\FunctionTok{GPCW}\NormalTok{(}\AttributeTok{mcmc\_samples =} \DecValTok{10000}\NormalTok{,}
              \AttributeTok{y =}\NormalTok{ y, }\AttributeTok{x =}\NormalTok{ x, }\AttributeTok{z =}\NormalTok{ z,}
              \AttributeTok{metrop\_var\_phi\_trans =} \FloatTok{1.15}\NormalTok{,}
              \AttributeTok{likelihood\_indicator =} \DecValTok{0}\NormalTok{)}
\end{Highlighting}
\end{Shaded}

\begin{verbatim}
## Progress: 10%
## phi Acceptance: 30%
## *******************
## Progress: 20%
## phi Acceptance: 29%
## *******************
## Progress: 30%
## phi Acceptance: 29%
## *******************
## Progress: 40%
## phi Acceptance: 28%
## *******************
## Progress: 50%
## phi Acceptance: 28%
## *******************
## Progress: 60%
## phi Acceptance: 28%
## *******************
## Progress: 70%
## phi Acceptance: 28%
## *******************
## Progress: 80%
## phi Acceptance: 28%
## *******************
## Progress: 90%
## phi Acceptance: 28%
## *******************
## Progress: 100%
## phi Acceptance: 28%
## *******************
\end{verbatim}

{[}3{]} Analyzing Output:

\begin{Shaded}
\begin{Highlighting}[]
\FunctionTok{par}\NormalTok{(}\AttributeTok{mfrow=}\FunctionTok{c}\NormalTok{(}\DecValTok{2}\NormalTok{,}\DecValTok{2}\NormalTok{))}
\FunctionTok{plot}\NormalTok{(results}\SpecialCharTok{$}\NormalTok{beta[}\DecValTok{1}\NormalTok{, }\DecValTok{1001}\SpecialCharTok{:}\DecValTok{10000}\NormalTok{], }
     \AttributeTok{type=}\StringTok{"l"}\NormalTok{,}
     \AttributeTok{ylab=}\StringTok{"beta"}\NormalTok{,}
     \AttributeTok{xlab=}\StringTok{"Sample"}\NormalTok{)}
\FunctionTok{abline}\NormalTok{(}\AttributeTok{h=}\NormalTok{beta\_true,}
       \AttributeTok{col=}\StringTok{"red"}\NormalTok{,}
       \AttributeTok{lwd=}\DecValTok{2}\NormalTok{)  }\CommentTok{\#True value}
\FunctionTok{plot}\NormalTok{(results}\SpecialCharTok{$}\NormalTok{sigma2\_theta[}\DecValTok{1001}\SpecialCharTok{:}\DecValTok{10000}\NormalTok{],}
     \AttributeTok{type=}\StringTok{"l"}\NormalTok{,}
     \AttributeTok{ylab=}\StringTok{"sigma2\_theta"}\NormalTok{,}
     \AttributeTok{xlab=}\StringTok{"Sample"}\NormalTok{)}
\FunctionTok{abline}\NormalTok{(}\AttributeTok{h=}\NormalTok{sigma2\_theta\_true,}
       \AttributeTok{col=}\StringTok{"red"}\NormalTok{,}
       \AttributeTok{lwd=}\DecValTok{2}\NormalTok{)  }\CommentTok{\#True value}
\FunctionTok{plot}\NormalTok{(results}\SpecialCharTok{$}\NormalTok{phi[}\DecValTok{1001}\SpecialCharTok{:}\DecValTok{10000}\NormalTok{],}
     \AttributeTok{type=}\StringTok{"l"}\NormalTok{,}
     \AttributeTok{ylab=}\StringTok{"phi"}\NormalTok{,}
     \AttributeTok{xlab=}\StringTok{"Sample"}\NormalTok{)}
\FunctionTok{abline}\NormalTok{(}\AttributeTok{h=}\NormalTok{phi\_true, }
       \AttributeTok{col=}\StringTok{"red"}\NormalTok{,}
       \AttributeTok{lwd=}\DecValTok{2}\NormalTok{)  }\CommentTok{\#True value}
\FunctionTok{plot}\NormalTok{(}\FunctionTok{rowMeans}\NormalTok{(results}\SpecialCharTok{$}\NormalTok{theta[,}\DecValTok{1001}\SpecialCharTok{:}\DecValTok{10000}\NormalTok{]), }
     \AttributeTok{pch=}\DecValTok{16}\NormalTok{,}
     \AttributeTok{ylab=}\StringTok{"theta"}\NormalTok{,}
     \AttributeTok{xlab=}\StringTok{"Time"}\NormalTok{)}
\FunctionTok{points}\NormalTok{(theta\_true, }
       \AttributeTok{pch=}\DecValTok{16}\NormalTok{, }
       \AttributeTok{col=}\StringTok{"red"}\NormalTok{)  }\CommentTok{\#True values}
\end{Highlighting}
\end{Shaded}

\includegraphics{GPCW_Example_files/figure-latex/unnamed-chunk-6-1.pdf}

\end{document}
